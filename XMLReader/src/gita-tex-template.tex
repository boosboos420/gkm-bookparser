\documentclass[a5paper,9pt,openany]{memoir}
\usepackage[top=.75in, bottom=.75in]{geometry}
\usepackage[T1]{fontenc}
\usepackage{palatino}
\usepackage{graphicx}
\usepackage{hyperref}
\usepackage{bookmark}

\setcounter{secnumdepth}{0}
\setcounter{tocdepth}{0}
\hypersetup{colorlinks, linkcolor=black}

\chapterstyle{dash}
% Generates the index
\usepackage{makeidx}
\makeindex


\newlength\drop
\makeatletter
\newcommand*\titleM{\begingroup% Misericords, T&H p 153
\setlength\drop{0.08\textheight}
\centering
\vspace*{\drop}
{\Huge\bfseries Journey Through The Bhagavad Gita}\\[\baselineskip]
{\scshape A Modern Commentary}\\[\baselineskip]
\vfill
{\large\scshape GK Marballi}\par
\vfill
{\scshape Azure Publishing}\par
{\scshape September 2013}\par
\vspace*{2\drop}
\endgroup}
\makeatother



\begin{document}

\pagestyle{simple}

\frontmatter

\begin{titlingpage}
\titleM
\newpage

% v.4 copyright page
~\vfill
Copyright \copyright\ \the\year\ GK Marballi. All rights reserved.\\
\par
ISBN 978-1-304-37595-7\\

\par Published by Azure Publishing\\

\par \textit{First printing, September 2013}\\
\end{titlingpage}

\newpage
\tableofcontents

% r.9 introduction
\newpage

%%
% Start the main matter (normal chapters)
\mainmatter

\chapter*{Introduction}
This book is a compilation of blog entries which were posted daily over a period of two years on http://journeygita.blogspot.com. Each blog post contains a commentary and translation of one verse of the Bhagavad Gita. Here is the entry which was posted on day one of the project:\\

*Om Ganeshaaya Namahaa*. Dedicated to all my teachers.\\

~\\I decided today to begin a blog that is my attempt to understand the Bhagavad Gita.\\

~\\I can't remember the exact date, but it was around 5 years ago that I chanced upon a television program in India where a prominent teacher of the Gita was conducting a discourse in English, stripped of all the usual pomp and ceremonial activities that usually accompany a discourse. As I watched that program, I realized that there is something here that is attracting me, but I could not quite explain why. I tried to watch that program as often as possible during my 3 month stay in India.\\

~\\When I returned to the US to continue my work, those Gita discourses stuck in my head. I began to buy commentaries on the Gita from several authors, and eventually came across a teacher whose audio discourses resonated with me. Now I am looking at the end of the 18th chapter (the last chapter) of the Gita, and having gone through it over the course of 2 years, still feel a need to understand it deeply.\\

~\\This blog is my attempt to restart my journey of the Gita, beginning with the first verse of chapter 1. But this time, I want to go deeper and understand each verse to the best of my ability. I will keep the language as simple as possible, and provide examples that I and most people can relate to.\\

\chapter*{About The Author}

GK Marballi works in the technology industry and is presently based in New York City. He received his bachelors degree in commerce from the University of Mumbai, and his MBA from Harvard Business School.\\

